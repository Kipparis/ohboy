\documentclass[12pt,a4paper,titlepage]{article}
\usepackage{exscale,latexsym,makeidx}

\usepackage[T2A,T1]{fontenc}
\usepackage[utf8]{inputenc}
\usepackage[russian]{babel}
\usepackage{amsmath}
\usepackage{amssymb}

% for indenting first line in section
\usepackage{indentfirst}

\author{Me}
\title{building myself}
\frenchspacing

\begin{document}
\maketitle \tableofcontents
\newpage
\section{aim}
\begin{itemize}
  \item Be the guy who has own IT shop and global website, in shop will sell my
custom builds.
  \item My name won't be well known. But my videos bout how to do different
things will.
  \item I will have nice low voice and nice appearence overall.
  \item I will have friends with those I can discuss future plans about
business.
  \item I will travel over countries to find the one, where I can live. With
most suitable people and government.
\end{itemize}

\newpage
\section{09.07.19}
\begin{center}
    \large Python
\end{center}
Today I learned how to \textbf{save} and \textbf{load} custom data type and way to store images.\\
Best practice is to place on first line assert line, then check passed values after: 
\begin{center}
    \begin{tabular}[t]{c | c | c}
        set failsafe values & raise an exception & leave as it is
    \end{tabular}
\end{center}
Learned the way to \textbf{unimplement} methods from class. 
Do it by raise Not\-Imp\-lement ( and hence TypeError or NotImplementError ) 

\begin{center}
    \large Latex 
\end{center}
Learned how to make table and floating objects.


\newpage
\section{10.07.19}
\begin{center}
    \large \textbf{Django}
\end{center}
Woke up as always, with dailty dozen and cold shower.\\
Then I knew how django rest api works:
\begin{itemize}
    \item{define model}
    \item{define serializer}
    \item{define use of serializer to show views}
    \item{make frontend use this data}
\end{itemize}
Most popular use: django $+$ react\\
How to wrap development and production of project.\\
All you need to do is: 
\begin{enumerate}
    \item{make bare repo}
    \item{clone it somewhere}
    \item{split settings (import with try catch)}
    \item{make Makefile for applying settings}
    \item{make post commit hook to restart server (for push fuctionality)}
\end{enumerate}

\begin{center}
    \large \textbf{Network}
\end{center}
Important to \textbf{understand what feels your companion}. Dialog is not about to insert your opinion.
You should feel the moment when insert a joke, or support morally, agree etc.\\
If it's your first meet, you may \textbf{show your insecurity}, show that companion can trust you.

\begin{center}
    \large \textbf{Javascript}
\end{center}
When js interpolates string, it goes left-to-right so result of next:
\begin{center}
    2 + 3 + "\ hi there " + 2 + 3
\end{center}
would be:
\begin{center}
    "5 hi there 23"
\end{center}

\newpage
\section{13.07.19}

\begin{center}
    \large \textbf{English}
\end{center}
For \textbf{pronouncing} best method for learning is shadow read. It's consists of:
\begin{enumerate}
    \item{Listening text}
    \item{Reading text}
    \item{Recording yourself}
    \item{Compare your pronouncing with author's}
\end{enumerate}
Another \textbf{common advices} for right pronouncing:
\begin{itemize}
    \item{not clean p,t,k but with noize}
    \item{v read as rus в, w read as rus уо (with opening mouth)}
    \item{th read with tongue between teeth}
    \item{ya instead of you}
\end{itemize}

\begin{center}
    \large \textbf{My}
\end{center}
Today made video about how to make vimtex setup for Latex development (on russian) becouse it doesn't exists yet.
Made a video how I drink shots.

\begin{center}
    \large \textbf{Django}
\end{center}
Finally install all dependencies and freeze it all, so its left to setup hooks and Makefile. \\
Afterwards I will make cleaner registering and personal area.

\newpage
\section{14.07.19}

\begin{center}
    \large \textbf{My}
\end{center}
Edit and post video about vimtex.\\
Thought that if I'll play dota 2, I would relax, but I wasted my time and didn't relax at all (4 games and only 1 win)\\
Structured notes in evernote.

\begin{center}
    \large \textbf{AI}
\end{center}
I had knew that to begin work with neural networks on c++ I have to know basics on python.\\
So I started to read book on DL on python.

\begin{center}
    \large \textbf{c++}
\end{center}
Nothing new.\\Last thing I read is about variable in scopes. So maybe I will skip some info.

\newpage
\section{15.07.19}

\begin{center}
    \large \textbf{Deep Learning}
\end{center}
I spent all day reading about neural networks. Now I understand how error is passed throw the neurons, how gradient descent works (and how I get to this).\\
Learned about:
\begin{itemize}
    \item{Matrixes}
    \item{Signals}
    \item{Values}
    \item{Error functions and gradient}
\end{itemize}
Knew that for beginners it's better to do ml and dl on python (becouse of speed of developing cycle), but when you got complite model, you may need to rewrite it on c++ for execution speed.

\begin{center}
    \large \textbf{Me}
\end{center}
Watched 2 series of adventure time for right pronounciation.

\newpage
\section{18.07.19}
\begin{center}
    \large \textbf{Deep Learning}
\end{center}

Made first neural network that recognizes numbers on 28 x 28 pic.

Learned about 
\begin{itemize}
    \item RNN (one input is nn on previous step like short term me\-mory)
    \item how translate different facts of world to neural network
\end{itemize} 

Knew about parameters of nn and how they affect. Like:
\begin{itemize}
    \item learning rate
    \item epoch num
    \item hidden layers num
    \item neurons for each hidden layer
\end{itemize}

Knew that I can change \textbf{activation function} to tanit and set -1 and 1.

\begin{center}
    \large \textbf{pwn}
\end{center}

Set up full \textbf{django push/deployment} script:

Onpush in automatically fetch changes to folder con\-ta\-ining si\-te so I don't ne\-ed to do enything except pushing changes to server (full local deve\-lop\-ment)

Nice collecting of staticfiles (even of base)

\textbf{Solved problem} with device width (very simple just set view\-port in base.html)

\newpage
\section{19.07.19}
\begin{center}
    \large \textbf{AI}
\end{center}

Today learned about tools that help to faster write AI bot for game (some of them are geneariled):
\begin{itemize}
    \item visualize - modify your environment to show how your bot takes desi\-cions
    \item move classes from previous project
    \item learn graph theory
    \item in last time check for other peoples' solutions and add if's to your code
\end{itemize}

My suggestions:
\begin{itemize}
    \item make fast template on python and improve in on c++
    \item use apropriate model (previous item)
\end{itemize}

New types AI that I learned:
\begin{enumerate}
    \item search algorithm - minimax --- MCTS
    \item local arena - fight against yourself
    \item see throw 50-100 most nearest actions and choose the best (first for enemy than for yourself)
    \item in case of use nn, make many simulations
\end{enumerate}

\begin{center}
    \large \textbf{pwn}
\end{center}

Developing own css for header.

Current plan is:
\begin{enumerate}
    \item make front page for desktop/tablet/phones and simple test within it
    \item registration throw all social media
    \item own area
    \item all for now, later see better
\end{enumerate}

\newpage
\section{20.07.19}
\begin{center}
    \large \textbf{dev tools}
\end{center}

Tune "go to definition" shortcut and start to learn vimtutor.

\newpage
\section{23.07.19}

I \textbf{realy} tired so here will be just few words about today.
\begin{center}
    \large \textbf{AI}
\end{center}

Learned how decrease dimension of data with math.

Started learning about probability theory and information theory. Ended on entropy.

Many new words and hard to realize them all in mind. But it's ok.

\begin{center}
    \large \textbf{me}
\end{center}

If you ignore people's q "why do I need it?", they'll ignore you.

\begin{center}
    \large \textbf{c++}
\end{center}

const, constexpr keyword - easy peasy

using, auto, decltype - same

started struct.

\newpage
\section{24.07.19}
\begin{center}
    \large \textbf{my}
\end{center}

\textbf{Best task strategy} - 45 min work - 15 min chill 

To best way to talk to other - tell only truth, be self confident, be lovely and friendly. Don't
be pussy and always hold your word. \textbf{Big impact} brings your voice:
\begin{itemize}
    \item make your voice more like "from bottom" it will cause the trust.
    \item don't let you sound boring: change pitch, tembr, rhime etc.
\end{itemize}

Make exersizes with your voice before important shows.

\begin{center}
    \large \textbf{c++}
\end{center}

\begin{itemize}
    \item struct
    \item while with cin and getline
    \item for (:) easy
    \item string shit like string::size\_type for storing size of string. also learned about how works conc strings
\end{itemize}

\newpage
\section{25.07.19}

\begin{center}
  \large \textbf{c++}
\end{center}

vector, iterators and operations on them

\begin{center}
  \large \textbf{pwn}
\end{center}

best practices in crawdfounding, mistakes, advantages, disadvantages all in evernote.
How to talk to other people and represent yourself. Be in theme.

\newpage
\section{26.07.19}

\begin{center}
  \large \textbf{me}
\end{center}

When talking be loud and self confident. Let your speak be as strong and you are.

When you're assure someone use one of this tools:
\begin{itemize}
  \item compare what person already knows
  \item look at him and show you know what you're talking about
  \item eye contact
  \item tell him what he understand
\end{itemize}

Started reading Aristotel. He tells about how speakers, judges and obser\-vers behave. Tells that \textbf{word} is a root of all tasks. You won't rich something really important (no matter how good you are) if you bad at convintion. Tell lots about lows, different roots of science and speaking, different goals for all peoples ( you won't agree that you were wrong is that were part of your work ). Define healthy status for people of all ages.\\
Well it's interesting. Goin' to read tomorrow.

\begin{center}
  \large \textbf{c++}
\end{center}

arrays and pointers to arrays. not hard at all, just refresh memory.

\newpage
\section{12.10.19}

\begin{center}
  \large \textbf{Tomorow plans}
\end{center}

\begin{itemize}
  \item 5:00 wake up and taking breakfast, speed up, shower, news
  \item 6:00 creating aim, 6:15 blind typing, 6:30 english, 6:45 voice
  \item 7:00 c++
  \item 8:00 cormen
  \item 9:00 python
  \item 10:00 website and searching for a teamates
  \item 11:00-17:00 vus
  \item 17:00 homework
  \item 18:00 reading literature for tomorrow day
  \item 20:00 free time (do lab, or music detector, movie or something)
  \item 22:00 sleep
\end{itemize}

\newpage
\section{14.10.19}

\begin{center}
  \large \textbf{What I done}
\end{center}

I answered well on today's english. (was training hard) \par
On morning I dir all according to plan, but waked up 1 hour later so
something I was to miss. \par
At c++ I learned about \textbf{mutable} objects that could change inside
of const member functions. \par
At python hour I learned about multiprogramming basics (not really
harder than c's fork xD) \par
But at home it was really dificult to study, maybe becouse I bought so
many shit products and I couldn't stop to ate them. Was vaping at home.
So next time I will not do this again.

\begin{center}
  \large \textbf{Tomorrow plans}
\end{center}

\begin{itemize}
  \item 5:00 wake up, shower, breakfast, news, speed up
  \item 6:00 reading aim, blind typing, english
  \item 7:00 c++
  \item 8:00 ai
  \item 8:30 go to vus
  \item 15-16 go for adruino parts
  \item later i am going to do python (1 hour), self education (1 hour),
    comple\-ting my tasks (1 hour)
\end{itemize}

\newpage
\section{21.10.19}

\textit{I learned that as I start to imagine what I will be able to do in the future, I am completely woke up}

\newpage
\section{10.11.19}
\textit{Hacaton's day}  
\begin{center}
  \textbf{Rules for make hackathon product faster}
\end{center}
\begin{enumerate}
  \item first decide whether task can be done with existing tools
  \item if not, split it
  \item go to first step
  \item if hackathon is 2-day long, don't sleep
  \item buy splitter
  \item ask best suite for product
\end{enumerate}
Several rules for presentation:
\begin{itemize}
  \item Don't repeat what is writen on slide (your slides must advance
    your wards or set the beginning or maybe if it's to long and boring
    to say)
  \item Be in move on scene (dont stay in one place, make active head
    turn, hand-play)
  \item Smile or make face simpler
  \item Be sure at what you say
  \item Increase or decrease voice volume and pitch. Change tempo and so
    on,
  \item Don't repeat your words and don't go to previous slides
  \item Don't use parasit-words
  \item Show that you can do many things
  \item At the end say something to make logic end
\end{itemize}
\textbf{After all this rules} your spich may become dry, so recommend you to
remember all stuff you want to say and ensert memes or jokes.

\begin{center}
  \textbf{After all}
\end{center}
Minus one task to do. Have stickers and expierence.

\newpage
\section{12.11.19}
\begin{center}
  \textbf{B-trees}
\end{center}
How i can insert in one top-down walk and without it. (just split nodes
with 2t-1 elements)
\begin{center}
  \textbf{tomorrow plans} 
\end{center}
\begin{enumerate}
  \item woke up at 6am
  \item play and eat up to 7am
  \item do all morning stuff up to 8am
  \item for the rest of the day: complete lp lab and do my own stuff.
  \item \textit{as bonus, read math to understand what is going on, for
    example should start with preparing for exames and learn what i
  don't understand}
\end{enumerate}
\par Good luck to tomorrow me xD
\par \textit{Hope I'll have something to write on next section...}

\newpage
\section{13.11.19}
\textit{again go to sleep at 1:00 :sad:}
\begin{center}
  \textbf{learned}
\end{center}
As it said, there no reason to implement of bycicle.\par
I'm not interested to understand full lp task, and spent just on it one
day expirementing with decisions what they mean of under each sentence.
So I just looked at answer and found what I need to do.\par
\textbf{Also} it's really good lifehack to play games at the morning, so
I tune myself on the day (as I know online games may affect in diffent
ways)
\begin{center}
  \textbf{Tomorrow plans}
\end{center}
Go again as yesterday, but with condition that I go to sleep at 9:30 and
read a book be4 go asleep.\par
learn how to make alarm without waking up all around.
\par Again, \textbf{I need to develop myself}

\newpage
\section{01.12.19}
\begin{center}
  \textbf{programming}
\end{center}
So many stuff to learn like shared libraries in os, or solution tree:
dfs, wfs, ids in lp.
\par I wanted to wake up earlier to stream and do my morning exersizes,
but go to sleep to late for this and, besides that, I'm ill. So woke up
at 11 am.

\begin{center}
  \textbf{archivements}
\end{center}
First day without chocolate, like back in past. Tomorrow I'll try to live
without nicotine, although tomorrow is vus.
\par Weekend is gone, but I don't feel myself rested. Maybe that's
becouse I go out from comforte zone and failed. But, in general, this
day was good. Like I do all at the morning and just was shopping all
day.

\begin{center}
  \textbf{project}
\end{center}
I have to recearch, why aplay can't go at the same time as chromium
plays music, and visa versa. If I discover the flow, it will be much
easear to make music playing.
\par Ok then, gone to read the book, cy@...

\newpage
\section{26.12.19}
\begin{center}
  \textbf{Tomorrow plans}
\end{center}
\begin{enumerate}
  \item Wake up 1 hour
    \subitem If it's hard go and buy energy drink
  \item Self development for 2 hours. possible themes for think:
    \subitem python linker and continue build app
    \subitem fin and invest from my dia and books
    \subitem english
    \subitem voice and health
  \item lab6 os
  \item kp
\end{enumerate}

\end{document}
